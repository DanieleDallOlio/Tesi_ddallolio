\chapter{Conclusioni}
I risultati esposti nel capitolo \ref{cap:3} sono le informazioni basilari per lo sviluppo futuro di una tecnica adeguata per la parallelizzazione dei processi.
Ciascuna caratteristica studiata mostra le potenzialità di un singolo nodo di un cluster definendone il singolo andamento e confrontandolo con gli altri presenti.

Tutte le analisi svolte sono state conseguite selezionando solamente una piccola frazione delle letture in possesso, spingendosi al massimo intorno al $6.7\%$. 
Infatti il numero di letture complessivo era ben oltre i 45 milioni e la massima grandezza di un subset condivisa tra tutti i dispositivi è stata di 3 milioni.

Solamente per il nodo più performante, con cpu Xeon, è stato deciso di spingersi oltre arrivando ai 9 milioni di letture contenute in un subset, che equivale a circa il $20\%$. 

Le conclusioni..

Questa tesi ha avuto lo scopo di riunire tutti gli elementi che hanno partecipato allo studio dell'analisi computazionale biofisica su nodi di macchine a basso consumo energetico, per preparare le basi ad un importante sviluppo futuro.  
Le prossime fasi che saranno coinvolte nella ricerca infatti, necessitano di un'iniziale analisi sui componenti computazionali più fondamentali perchè vengano fornite le informazioni per programmare una tecnologia efficace, fruibile ed economica.

Il percorso di studio futuro avrà come primo step il confronto dei dati con un modello di macchina tradizionale, non low power, tra quelle utilizzate quotidianamente.
In seguito saranno adattate le simulazioni all'utilizzo contemporaneo di tutti i core previsti dalle macchine, visto che solo uno step del procedimento è stato trattato con un core aggiuntivo(il mapping).
Infine le analisi verranno svolte, dopo un'opportuna progettazione, su un gruppo di cluster low power per verificare concretamente che l'utilizzo di un insieme cooperante di apparecchi a basso consumo energetico è effettivamente performante almeno quanto una macchina tradizionale.

L'esito positivo della ricerca consentirebbe agli enti accademici e ospedialieri di adottare una tecnologia più accessibile, meno costosa e soprattutto più veloce, che sostenga la ricerca in medicina e in biologia, oltre che il generale progresso scientifico.


