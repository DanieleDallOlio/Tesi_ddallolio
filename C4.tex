\chapter{Conclusioni}
Il lavoro effettuato durante lo svolgimento di questa tesi ha lo scopo di esplorare la possibilità di sostituire i nodi di calcolo tradizionalmente usati per calcoli ad alta performance, con nodi a basso consumo energetico.
Questo è motivato dalla possibilità di avere una maggior potenza di calcolo a parità di spesa, grazie al minor costo per unità dei server low power, a fronte di una capacità di calcolo comparabile.
Per testare questi nodi di calcolo è stata sfruttata una pipeline di calcolo bioinformatico, che ha requisiti molto elevati in termini di potenza di calcolo, di occupazione di memoria e di spazio d'archiviazione.

Questa pipeline è stata testata su quattro nodi dai requisiti di potenza progressivamente più elevati, partendo da una macchina a consumi molto ridotti (6W per n3700) fino ad un nodo di calcolo tradizionale (180W per bio8).
Le analisi delle simulazioni effettuate, riportate nel capitolo \ref{cap:3}, hanno approfondito il tempo di esecuzione e la quantità di memoria occupata per ciascuna operazione svolta nella pipeline.
Questi due parametri sono fondamentali per la progettazione dei cluster di calcolo in quanto determinano la dimensione minima necessaria dei nodi.

L'analisi dei tempi di esecuzione mostra come vi siano due gruppi di nodi di calcolo distinguibili in base all'architettura del processore.
I due nodi a potenza più ridotta (n3700 e avoton) hanno dei tempi d'esecuzione generalmente doppi rispetto a quelli a potenza più elevata.
Il terzo nodo(xeond), sempre low power, ha dei tempi di esecuzione confrontabili con il nodo di calcolo tradizionale(bio8) a fronte di un consumo energetico di circa un terzo e di un costo dieci volte inferiore.

L'analisi della memoria occupata dalle varie regole mostra due diversi comportamenti.
In un caso, si ha una saturazione della una quantità di memoria occupata ad un valore fisso, che varia poco tra i diversi nodi.
Nell'altro caso le regole si adattano dinamicamente alla quantità di memoria libera disponibile.
In entrambi i casi la memoria occupata dalle singole regole è sempre risultata inferiore alla memoria disponibile in tutti i nodi, occupando al massimo 5\,GB a fronte di una memoria complessiva di almeno 8\,GB.

In base a questi risultati questa pipeline di calcolo bioinformatico sembra essere realisticamente eseguibile anche su nodi a bassa potenza senza una perdita considerevole di prestazioni.
Questo, combinato con la possibilità di suddividere il calcolo in maniera \textit{embarassingly parallel} rende i nodi di calcolo low power un'alternativa concreta a quelli tradizionali. 

Il percorso di studio futuro avrà come primo step la ripetizione delle simulazioni sui singoli nodi sfruttando più core. 
Successivamente sarà completata la pipeline di GATK-LOD\ped{n} e saranno svolte nuove simulazioni per confermare i risultati ottenuti in precedenza.
In seguito saranno effettuate nuove simulazioni su interi cluster, sia low power che tradizionali che cloud.

L'esito positivo della ricerca consentirebbe agli enti accademici e ospedialieri di adottare una tecnologia più accessibile, meno costosa e soprattutto più veloce, che sostenga la ricerca in medicina e in biologia.


