\documentclass{beamer}
\usepackage[english,italian]{babel}
\usepackage{booktabs}
\usepackage{listings}
\usepackage[utf8]{inputenc}
\usepackage{amsmath}
\usepackage{float}
\usepackage{eurosym}
\usepackage{hyperref}
\usepackage[scientific-notation=true]{siunitx}
\usepackage{wrapfig}
\usepackage{subfig}
\usepackage{cite}
\usepackage{capt-of}
\usepackage{booktabs, caption, makecell}
\renewcommand\theadfont{\bfseries}
\usepackage{threeparttable}
\usepackage{varwidth}
\usepackage{tabularx,colortbl}
\usepackage{siunitx}
\usepackage{pdfpages}
\usepackage{color}

\graphicspath{{./Figure/} {./Figure/Grafici_Finali/}}

\title[]{\textbf{Implementazione, creazione e ottimizzazione di una pipeline per l'analisi biofisica su cluster a basso consumo energetico}}
\author[Daniele Dall'Olio]{Daniele Dall'Olio\\{\small Relatore: Dott. Enrico Giampieri \\ Correlatori: Prof. Gastone Castellani \and Ing. Andrea Ferraro}}
\date{22 Settembre 2017}
\institute[]{ALMA MATER STUDIORUM $\cdot$ UNIVERSIT\'A DI BOLOGNA}

\usetheme{CambridgeUS}
\setbeamertemplate{blocks}[rounded][shadow=true]

\begin{document}

\begin{frame}
\maketitle
\end{frame}

\section{Introduzione}

%\begin{frame}
%\begin{columns}
%\begin{column}{0.8\linewidth}
%\begin{block}{Analisi Dati in biomedicina}
%\begin{itemize}
%\item Big Data
%\item Elaborazioni complesse
%\item Teoria dei Network
%\end{itemize}
%\end{block}
%\begin{block}{Richiesta}
%Elevata potenza di calcolo
%\end{block}
%\begin{block}{Strumentazione comune}
%Macchine tradizionali ad alta performance
%\end{block}
%\end{column}
%\end{columns}
%\end{frame}

\subsection{Difetti delle macchine tradizionali}

\begin{frame}
\begin{columns}
\begin{column}{0.8\linewidth}
			
\begin{block}{Problema}
\begin{itemize}
\item Costo medio elevato
\item Consumo energetico elevato
\item Spese per il raffreddamento elevate
\end{itemize}
\begin{block}{Conseguenze}
\begin{itemize}
\item Minor accessibilità
\item Poche unità acquistabili
\item Ridotta scalabilità e flessibilità per aggiornare l’hardware dei server
\end{itemize}
\end{block}
\end{block}
\end{column}
\end{columns}
\end{frame}	

\subsection{Metodo Alternativo}
\begin{frame}
\begin{columns}
\begin{column}{0.8\linewidth}		
\begin{block}{Tecnologia di calcolo low power}
\begin{itemize}
\begin{block}{Vantaggi}
\small
\item Costo delle singole unità basso
\item Consumo elettrico inferiore
\item Flessibilità nell'acquisto di nuovi hardware
\end{block}
\begin{block}{Svantaggi}
\small
\item Cache ridotta
\item Potenza inferiore
\item Numero inferiori di core
\end{block}
\end{itemize}
\end{block}
\begin{block}{Obiettivo}
\small
\textbf{Ottenere risultati
comparabili a quelli ottenuti con i nodi tradizionali.}
\end{block}
\end{column}
\end{columns}
\end{frame}

\subsection{I nodi utilizzati}
\begin{frame}
\begin{columns}
\begin{column}{0.8\linewidth}
\begin{table}[H]
\begin{threeparttable}
\resizebox{1.0\textwidth}{!}{%
$\begin{array}{*{6}{c}}
	\toprule
		Nodo & CPU & Memory & Storage & Costo\text{*} & Consumo\text{*}  \\
	\midrule
		xeond & \text{1x Xeon D-1540} & 16\,GB & 8\,TB(HDD) & \text{\euro 1000} & 60\,W\\
		avoton & \text{1x Atom C2750}  & 16\,GB & 5\,TB(HDD) & \text{\euro 600} & 30\,W\\
		n3700 & \text{1x Pentium N3700}  & 8\,GB & 0.5\,TB(SSD) & \text{\euro 130} & 8\,W \\
	\midrule
	\rowcolor{yellow}		
		bio8 & \text{2x Xeon E5-2620v4} & 128\,GB & 2\,TB(HDD) & \text{\euro 10000} & 180\,W\\		
	\bottomrule
\end{array}$%
}
\begin{tablenotes}\footnotesize
\item[*] I valori di costo e consumo energetico sono stimati.
\end{tablenotes}
\end{threeparttable}
\caption{Caratteristiche dei nodi.}
\label{tab:cluster_generali}
\end{table}

\begin{table}[H]
\centering
\resizebox{1.0\textwidth}{!}{%
$\begin{array}{*{6}{c}}
	\toprule
		CPU & Microarchitecture(Platform)/litho & Freq(GHz) & Cores & Cache & TDP\\
	\midrule
		\text{Xeon D-1540} & Broadwell/14nm & 2.0(2.60) & 8(16) & 12\,MB & 45\,W \\
		\text{Atom C2750} & Silvermont(Avoton)/22nm & 2.40(2.60) & 8 & 4\,MB & 25\,W \\
		\text{Pentium N3700} & Airmont(Braswell)/14nm & 1.60(2.40) & 4 & 2\,MB & 6\,W \\
	\midrule
	\rowcolor{yellow}	
		\text{Xeon E5-2620v4} & Broadwell-EP/14nm & 2.10(3.00) & 8(16) & 20\,MB & 85\,W \\
	\bottomrule
\end{array}$%
}
\caption{Caratteristiche delle CPU.}
\label{tab:cpu}
\end{table}
\end{column}
\end{columns}
\end{frame}

\subsection{Pipeline di calcolo bioinformatico}
\begin{frame}
\begin{columns}
\begin{column}{0.8\linewidth}
\begin{block}{GATK-LODn}
Requisiti molto elevati in termini di potenza di calcolo, di occupazione di memoria e di spazio d’archiviazione.
\begin{figure}
\centering
\includegraphics[scale=0.3]{GATK-LODn.png}
\end{figure}
\end{block}
\end{column}
\end{columns}
\end{frame}

\subsection{NGS}
\begin{frame}
\begin{columns}
\begin{column}{0.8\linewidth}
\begin{block}{NGS}
\begin{itemize}
\small
\item Comprende le nuove tecniche per il sequenziamento del 
DNA
\item Succede al HGP e supera il metodo Sanger
\item Tecniche più rapide e meno costose
\item Utilizzo della Teoria dei Network
\item \textbf{Shotgun Sequencing}
\end{itemize}
\end{block}
\end{column}
\end{columns}
\end{frame}


\section{Materiali e Metodi}

\subsection{Struttura delle simulazioni}
\begin{frame}
\begin{columns}
\begin{column}{0.8\linewidth}	
\begin{block}{Struttura delle simulazioni}
Una parte di GATK-LOD\ped{n} è stata reimplementata nel tool Snakemake.
\begin{figure}[H]
\centering
\includegraphics[scale=0.3]{Workflow.jpg}
\end{figure}
\end{block}
\end{column}
\end{columns}
\end{frame}

\subsection{La parte considerata}
\begin{frame}
\begin{columns}
\begin{column}{0.8\linewidth}
\begin{block}{Regole}
\begin{block}{Indipendenti dal paziente}
\begin{itemize}
\small
\item Indicizzazione per BWA
\item Indicizzazione per Picard
\item Indicizzazione per Samtools(e GATK) 
\end{itemize}
\begin{block}{Dipendenti dal paziente}
\small
\begin{itemize}
\small
\item \textbf{Mapping}: mappatura delle sequenze del paziente sul riferimento(in SAM).
\item \textbf{Sort Picard}: riordinamento dei file SAM(in BAM). 
\item \textbf{Mark Duplicates}: identificazione dei duplicati.
\item \textbf{Build BAM}: indicizza il file BAM per velocizzare l'analisi.
\item \textbf{Realigner}: determina gli intervalli che necessitano probabilmente del riallineamento Indel.
\end{itemize}
\end{block}
\end{block}
\end{block}
\end{column}
\end{columns}
\end{frame}


\subsection{Analisi statistiche}
\begin{frame}
\begin{columns}
\begin{column}{0.8\linewidth}	
\begin{block}{Analisi effettuate}
\begin{itemize}
\item Tempo di esecuzione
\item Memoria utilizzata
\end{itemize}
\end{block}
\begin{block}{Simulazioni effettuate}
\begin{table}[H]
	\centering
	\begin{tabular}{lr}
		\toprule
			\text{numero di letture} & \text{dimensione su disco} \\
		\midrule
			\num{1e5}   & \text{2x 28.4\,MB} \\
			\num{1e6}     & \text{2x 284.9\,MB} \\
			\num{3e6}     & \text{2x 854.9\,MB} \\
			\num{9e6}     & \text{2x 2.6\,GB} \\
		\midrule
		\rowcolor{yellow}		
			\num{4.5e7}    & \text{2x 12.8\,GB} \\
		\bottomrule
	\end{tabular}
	\caption{Stima della dimensione dei subset in relazione al numero di letture. L'ultimo valore si riferisce all'intero paziente.}
\end{table}
\end{block}
\end{column}
\end{columns}
\end{frame}

\section{Risultati}
\subsection{Tempo di esecuzione}
\begin{frame}
\begin{columns}
\begin{column}{0.8\linewidth}	
\begin{figure}[H]
\centering
\includegraphics[scale=0.5]{mapping.png}	
\captionof{figure}{Tempi per Mapping.}
\label{subfig:Map}
\end{figure}
\end{column}
\end{columns}
\end{frame}

\begin{frame}
\begin{columns}
\begin{column}{0.8\linewidth}	
\begin{figure}[H]
\centering
\includegraphics[scale=0.5]{sort_picard.png}
\captionof{figure}{Tempi per Sort Picard.}
\label{subfig:SP}
\end{figure}
\end{column}
\end{columns}
\end{frame}

\begin{frame}
\begin{columns}
\begin{column}{0.8\linewidth}	
\begin{figure}[H]
\centering
\includegraphics[scale=0.43]{Tempi_complessivi.png}
\captionof{figure}{Tempi complessivi.}
\label{subfig:SP}
\end{figure}
\end{column}
\end{columns}
\end{frame}


\subsection{Memoria utilizzata}
\begin{frame}
\begin{columns}
\begin{column}{0.8\linewidth}	
\begin{figure}[H]
\centering
\includegraphics[scale=0.46]{Max_rss_mapping.png}
\caption{Mapping.}
\label{fig:RSSind}
\end{figure}
\end{column}
\end{columns}
\end{frame}

\begin{frame}
\begin{columns}
\begin{column}{0.8\linewidth}	
\begin{figure}[H]
\centering
\includegraphics[scale=0.46]{Max_rss_sort_picard.png}
\caption{Sort Picard.}
\label{fig:RSSind}
\end{figure}
\end{column}
\end{columns}
\end{frame}

\begin{frame}
\begin{columns}
\begin{column}{0.8\linewidth}	
\begin{figure}[H]
\centering
\includegraphics[scale=0.46]{Max_rss_realigner.png}
\caption{Realigner.}
\label{fig:RSSind}
\end{figure}
\end{column}
\end{columns}
\end{frame}

\section{Conclusioni}
\begin{frame}
\begin{columns}
\begin{column}{0.8\linewidth}	
\begin{block}{Tempo di esecuzione}
\begin{itemize}
\item avoton e n3700 impiegano il doppio del tempo
\item xeond è comparabile a bio8 consumando un terzo dell'energia e costando 10 volte di meno
\end{itemize}
\end{block}
\begin{block}{Memoria utilizzata}
\begin{itemize}
\item Saturazione
\item Adattamento dinamico
\item Sempre inferiore al massimo di memoria accessibile
\end{itemize}
\end{block}
\begin{block}{Conclusione}
\textbf{In base a questi risultati questa pipeline di calcolo bioinformatico sembra
essere realisticamente eseguibile anche su nodi a bassa potenza senza una
perdita considerevole di prestazioni.}
\end{block}
\end{column}
\end{columns}
\end{frame}

\subsection{Sviluppo futuro}
\begin{frame}
\begin{columns}
\begin{column}{0.8\linewidth}	
\begin{block}{Sviluppo futuro}
\begin{itemize}
\item Simulazioni a core multipli sui singoli nodi
\item Completamento della pipeline
\item Simulazioni su cluster
\end{itemize}
\end{block}
\begin{block}{Pubblicazione}
Cercheremo di completare il progetto e infine di pubblicarlo.
\end{block}
\end{column}
\end{columns}
\end{frame}

\end{document}

